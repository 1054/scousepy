%% Generated by Sphinx.
\def\sphinxdocclass{report}
\documentclass[letterpaper,10pt,english]{sphinxmanual}
\ifdefined\pdfpxdimen
   \let\sphinxpxdimen\pdfpxdimen\else\newdimen\sphinxpxdimen
\fi \sphinxpxdimen=.75bp\relax

\PassOptionsToPackage{warn}{textcomp}
\usepackage[utf8]{inputenc}
\ifdefined\DeclareUnicodeCharacter
% support both utf8 and utf8x syntaxes
\edef\sphinxdqmaybe{\ifdefined\DeclareUnicodeCharacterAsOptional\string"\fi}
  \DeclareUnicodeCharacter{\sphinxdqmaybe00A0}{\nobreakspace}
  \DeclareUnicodeCharacter{\sphinxdqmaybe2500}{\sphinxunichar{2500}}
  \DeclareUnicodeCharacter{\sphinxdqmaybe2502}{\sphinxunichar{2502}}
  \DeclareUnicodeCharacter{\sphinxdqmaybe2514}{\sphinxunichar{2514}}
  \DeclareUnicodeCharacter{\sphinxdqmaybe251C}{\sphinxunichar{251C}}
  \DeclareUnicodeCharacter{\sphinxdqmaybe2572}{\textbackslash}
\fi
\usepackage{cmap}
\usepackage[T1]{fontenc}
\usepackage{amsmath,amssymb,amstext}
\usepackage{babel}
\usepackage{times}
\usepackage[Bjarne]{fncychap}
\usepackage{sphinx}

\fvset{fontsize=\small}
\usepackage{geometry}

% Include hyperref last.
\usepackage{hyperref}
% Fix anchor placement for figures with captions.
\usepackage{hypcap}% it must be loaded after hyperref.
% Set up styles of URL: it should be placed after hyperref.
\urlstyle{same}

\addto\captionsenglish{\renewcommand{\figurename}{Fig.\@ }}
\makeatletter
\def\fnum@figure{\figurename\thefigure{}}
\makeatother
\addto\captionsenglish{\renewcommand{\tablename}{Table }}
\makeatletter
\def\fnum@table{\tablename\thetable{}}
\makeatother
\addto\captionsenglish{\renewcommand{\literalblockname}{Listing}}

\addto\captionsenglish{\renewcommand{\literalblockcontinuedname}{continued from previous page}}
\addto\captionsenglish{\renewcommand{\literalblockcontinuesname}{continues on next page}}
\addto\captionsenglish{\renewcommand{\sphinxnonalphabeticalgroupname}{Non-alphabetical}}
\addto\captionsenglish{\renewcommand{\sphinxsymbolsname}{Symbols}}
\addto\captionsenglish{\renewcommand{\sphinxnumbersname}{Numbers}}

\addto\extrasenglish{\def\pageautorefname{page}}

\setcounter{tocdepth}{3}
\setcounter{secnumdepth}{3}


\title{scousepy Documentation}
\date{Feb 11, 2019}
\release{0.0.1.dev}
\author{Jonathan D. Henshaw}
\newcommand{\sphinxlogo}{\vbox{}}
\renewcommand{\releasename}{Release}
\makeindex
\begin{document}

\pagestyle{empty}
\sphinxmaketitle
\pagestyle{plain}
\sphinxtableofcontents
\pagestyle{normal}
\phantomsection\label{\detokenize{index::doc}}
\noindent{\hspace*{\fill}\sphinxincludegraphics[width=750\sphinxpxdimen]{{SCOUSE_LOGO}.png}\hspace*{\fill}}

\begin{DUlineblock}{0em}
\item[] 
\end{DUlineblock}



The \sphinxcode{\sphinxupquote{scousepy}} package provides a method by which a large amount of complex
astronomical spectral line data can be fitted in a systematic way. A detailed
description of the method (and the original \sphinxhref{https://github.com/jdhenshaw/SCOUSE}{IDL version} of the code)
can be found in \sphinxhref{http://ukads.nottingham.ac.uk/abs/2016arXiv160103732H}{Henshaw et al. 2016}.
In the following pages you will find a {\hyperref[\detokenize{description:introduction}]{\sphinxcrossref{\DUrole{std,std-ref}{brief introduction}}}}
to the method as well as a {\hyperref[\detokenize{tutorial:tutorial}]{\sphinxcrossref{\DUrole{std,std-ref}{tutorial}}}}. The \sphinxhref{https://github.com/jdhenshaw/scousepy}{source code} is available on GitHub and comments
and contributions are very welcome.


\chapter{Documentation}
\label{\detokenize{index:documentation}}

\section{Installing \sphinxstyleliteralintitle{\sphinxupquote{scousepy}}}
\label{\detokenize{installation:installing-scousepy}}\label{\detokenize{installation::doc}}

\subsection{Requirements}
\label{\detokenize{installation:requirements}}
\sphinxcode{\sphinxupquote{scousepy}} requires the following packages:
\begin{itemize}
\item {} 
\sphinxhref{http://www.python.org}{Python} 3.x

\item {} 
\sphinxhref{http://www.astropy.org/}{astropy} \textgreater{}=3.0.2

\item {} 
\sphinxhref{http://lmfit.github.io/lmfit-py/}{lmfit} \textgreater{}=0.8.0

\item {} 
\sphinxhref{http://matplotlib.org/}{matplotlib} \textgreater{}=2.2.2

\item {} 
\sphinxhref{http://www.numpy.org/}{numpy} \textgreater{}=1.14.2

\item {} 
\sphinxhref{http://pyspeckit.readthedocs.io/en/latest/}{pyspeckit} \textgreater{}=0.1.21.dev2682

\item {} 
\sphinxhref{http://spectral-cube.readthedocs.io/en/latest/}{spectral\_cube} \textgreater{}=0.4.4.dev1809

\end{itemize}

Note that for interactive fitting with pyspeckit you may need to customise your
matplotlib configuration. Namely, if you’re using \sphinxcode{\sphinxupquote{scousepy}} on a Mac you will
most likely need to change your backend from ‘macosx’ to ‘Qt5Agg’ (or equiv.).
You can find some information about how to do this \sphinxhref{https://matplotlib.org/users/customizing.html\#customizing-matplotlib}{here}.


\subsection{Installation}
\label{\detokenize{installation:installation}}
(Available soon - stick to developer version for now - see below)

To install the latest stable release, you can type:

\begin{sphinxVerbatim}[commandchars=\\\{\}]
\PYG{n}{pip} \PYG{n}{install} \PYG{n}{scousepy}
\end{sphinxVerbatim}

or you can download the latest tar file from
\sphinxhref{https://pypi.python.org/pypi/scousepy}{PyPI} and install it using:

\begin{sphinxVerbatim}[commandchars=\\\{\}]
\PYG{n}{python} \PYG{n}{setup}\PYG{o}{.}\PYG{n}{py} \PYG{n}{install}
\end{sphinxVerbatim}


\subsection{Developer version}
\label{\detokenize{installation:developer-version}}
If you want to install the latest developer version of the scousepy, you can do
so from the git repository:

\begin{sphinxVerbatim}[commandchars=\\\{\}]
\PYG{n}{git} \PYG{n}{clone} \PYG{n}{https}\PYG{p}{:}\PYG{o}{/}\PYG{o}{/}\PYG{n}{github}\PYG{o}{.}\PYG{n}{com}\PYG{o}{/}\PYG{n}{jdhenshaw}\PYG{o}{/}\PYG{n}{scousepy}
\PYG{n}{cd} \PYG{n}{scousepy}
\PYG{n}{python} \PYG{n}{setup}\PYG{o}{.}\PYG{n}{py} \PYG{n}{install}
\end{sphinxVerbatim}

You may need to add the \sphinxcode{\sphinxupquote{-{-}user}} option to the last line if you do not have
root access.


\section{A brief introduction to scousepy}
\label{\detokenize{description:a-brief-introduction-to-scousepy}}\label{\detokenize{description:introduction}}\label{\detokenize{description::doc}}
The method has been updated slightly from the \sphinxhref{https://github.com/jdhenshaw/SCOUSE}{original IDL version of the
code}. It is now more interactive than
before which should hopefully speed things up a bit for the user. The method
is broken down into six stages in total. Each stage is summarised below.

\noindent{\hspace*{\fill}\sphinxincludegraphics[width=850\sphinxpxdimen]{{Figure_cartoon}.png}\hspace*{\fill}}


\subsection{Stage 1: defining the coverage}
\label{\detokenize{description:stage-1-defining-the-coverage}}
Here \sphinxcode{\sphinxupquote{scousepy}} identifies the spatial area over which to fit the data. It
generates a grid of spectral averaging areas (SAAs). The user is required to
provide the width of the spectral averaging area. Extra refinement of spectral
averaging areas (i.e. for complex regions) can be controlled using the keyword
\sphinxtitleref{refine\_grid}.


\subsection{Stage 2: fitting the spectral averaging areas}
\label{\detokenize{description:stage-2-fitting-the-spectral-averaging-areas}}
User-interactive fitting of the spatially averaged spectra output from stage 1.
\sphinxcode{\sphinxupquote{scousepy}} makes use of the \sphinxhref{http://pyspeckit.readthedocs.io/en/latest/}{pyspeckit}
package and is fully interactive.


\subsection{Stage 3: automated fitting}
\label{\detokenize{description:stage-3-automated-fitting}}
Non user-interactive fitting of individual spectra contained within all SAAs.
The user is required to input several tolerance levels to \sphinxcode{\sphinxupquote{scousepy}}. Please
refer to \sphinxhref{http://adsabs.harvard.edu/abs/2016MNRAS.457.2675H}{Henshaw et al. 2016}
for more details on each of these.


\subsection{Stage 4: selecting the best fits}
\label{\detokenize{description:stage-4-selecting-the-best-fits}}
Here \sphinxcode{\sphinxupquote{scousepy}} selects the best-fits that are output in stage 3.


\subsection{Optional Stages}
\label{\detokenize{description:optional-stages}}
Unfortunately there is no one-size-fits-all method to selecting a best-fitting
solution when multiple choices are available (stage 4). SCOUSE uses the Akaike
Information Criterion, which weights the chi-squared value of a best-fitting
solution according to the number of free-parameters.

While AIC does a good job of returning the best-fitting solutions, there are
areas where the best-fitting solutions can be improved. As such the following
stages are optional but \sphinxstyleemphasis{highly recommended}.

This part of the process has changed significantly from the original code. The
user is now presented with several diagnostic plots (see below), selecting
different regions will display the corresponding spectra, allowing the user to
check the fit quality.

Depending on the data a user may wish to perform a few iterations of Stages 5-6.


\subsection{Stage 5: checking the best-fitting solutions}
\label{\detokenize{description:stage-5-checking-the-best-fitting-solutions}}
Checking the fits. Here the user is required to check the best-fitting
solutions to the spectra. This stage is now fully interactive. The user is first
presented with several diagnostic plots namely: \sphinxtitleref{rms}, \sphinxtitleref{residstd}, \sphinxtitleref{redchi2},
\sphinxtitleref{ncomps}, \sphinxtitleref{aic}, \sphinxtitleref{chi2}. These can be used to assess the quality of fits
throughout the map. Clicking on a particular region will show the spectra
associated with that location. The user can then select spectra for closer
inspection or refitting as required.


\subsection{Stage 6: re-analysing the identified spectra}
\label{\detokenize{description:stage-6-re-analysing-the-identified-spectra}}
In this stage the user is required to either select an alternative solution or
re-fit completely the spectra identified in stage 5.


\section{Tutorial}
\label{\detokenize{tutorial:tutorial}}\label{\detokenize{tutorial:id1}}\label{\detokenize{tutorial::doc}}

\subsection{Data}
\label{\detokenize{tutorial:data}}
This tutorial utilises observations of N2H+ (1-0) towards the Infrared Dark
Cloud G035.39-00.33. This data set was first published in \sphinxhref{http://adsabs.harvard.edu/abs/2013MNRAS.428.3425H}{Henshaw et al. 2013.}.
These observations were carried out with the IRAM 30m Telescope. IRAM is
supported by INSU/CNRS (France), MPG (Germany) and IGN (Spain).


\section{License}
\label{\detokenize{license:license}}\label{\detokenize{license::doc}}
MIT License

Copyright (c) 2017-2019 Jonathan D. Henshaw

Permission is hereby granted, free of charge, to any person obtaining a copy
of this software and associated documentation files (the “Software”), to deal
in the Software without restriction, including without limitation the rights
to use, copy, modify, merge, publish, distribute, sublicense, and/or sell
copies of the Software, and to permit persons to whom the Software is
furnished to do so, subject to the following conditions:

The above copyright notice and this permission notice shall be included in all
copies or substantial portions of the Software.

THE SOFTWARE IS PROVIDED “AS IS”, WITHOUT WARRANTY OF ANY KIND, EXPRESS OR
IMPLIED, INCLUDING BUT NOT LIMITED TO THE WARRANTIES OF MERCHANTABILITY,
FITNESS FOR A PARTICULAR PURPOSE AND NONINFRINGEMENT. IN NO EVENT SHALL THE
AUTHORS OR COPYRIGHT HOLDERS BE LIABLE FOR ANY CLAIM, DAMAGES OR OTHER
LIABILITY, WHETHER IN AN ACTION OF CONTRACT, TORT OR OTHERWISE, ARISING FROM,
OUT OF OR IN CONNECTION WITH THE SOFTWARE OR THE USE OR OTHER DEALINGS IN THE
SOFTWARE.


\chapter{Reporting issues and getting help}
\label{\detokenize{index:reporting-issues-and-getting-help}}
Please help to improve this package by reporting \sphinxhref{https://github.com/jdhenshaw/scousepy/issues}{issues} via GitHub. Alternatively, if
you have any questions or if you are having any problems getting set up you can
get in touch \sphinxhref{mailto:jonathan.d.henshaw@gmail.com}{here}.


\chapter{Developers}
\label{\detokenize{index:developers}}
This package was developed by:
\begin{itemize}
\item {} 
Jonathan Henshaw

\end{itemize}

Contributors include:
\begin{itemize}
\item {} 
Adam Ginsburg

\item {} 
Manuel Reiner

\end{itemize}


\chapter{Citing scousepy}
\label{\detokenize{index:citing-scousepy}}
If you make use of this package in a publication, please consider the following
acknowledgement…
\begin{quote}

Henshaw et al. 2018 (in prep. coming soon)
\end{quote}

Please also consider acknowledgements to the required packages in your work.


\chapter{Papers using scousepy}
\label{\detokenize{index:papers-using-scousepy}}\begin{itemize}
\item {} 
Henshaw et al. 2018, in prep.

\end{itemize}


\chapter{Recipe}
\label{\detokenize{index:recipe}}
Recipe for a fine Liverpudlian \sphinxhref{http://www.bbc.co.uk/food/recipes/scouse\_pie\_49004}{Scouse pie}.



\renewcommand{\indexname}{Index}
\printindex
\end{document}